\documentclass[conference]{IEEEtran}
\usepackage{graphicx}
\usepackage{url}
\usepackage{amsmath}

\begin{document}

\title{Predictive Modeling of Heart Disease Using BRFSS 2015 Dataset}

\maketitle

\begin{abstract}
This study presents a comprehensive analysis of heart disease prediction using the BRFSS 2015 dataset. We employed multiple machine learning approaches to identify key risk factors and develop predictive models for heart disease occurrence. The analysis includes data preprocessing, handling class imbalance, and evaluation of various classification models. Our findings indicate significant correlations between certain health indicators and heart disease, providing valuable insights for public health interventions.
\end{abstract}

\section{Introduction}
Heart disease remains a leading cause of mortality worldwide. Early detection and risk assessment are crucial for prevention and treatment. This study leverages machine learning techniques to analyze the Behavioral Risk Factor Surveillance System (BRFSS) 2015 dataset, aiming to develop accurate predictive models for heart disease occurrence.

\section{Methodology}
\subsection{Dataset Description}
The BRFSS 2015 dataset contains 253,680 observations with 22 features, including health indicators, demographic information, and lifestyle factors. The target variable is binary, indicating the presence or absence of heart disease or heart attack history.

\subsection{Data Preprocessing}
Key preprocessing steps included:
\begin{itemize}
    \item Handling missing values using median imputation
    \item Encoding categorical variables
    \item Scaling numerical features using StandardScaler
    \item Addressing class imbalance using SMOTE
\end{itemize}

\subsection{Model Development}
We implemented three classification models:
\begin{itemize}
    \item Logistic Regression
    \item Random Forest
    \item Support Vector Machine (SVM)
\end{itemize}

\section{Results and Discussion}
\subsection{Data Exploration Findings}
\begin{itemize}
    \item The dataset shows a significant class imbalance, with approximately 9.4\% of cases positive for heart disease
    \item Key risk factors identified through correlation analysis include:
    \begin{itemize}
        \item High blood pressure
        \item High cholesterol
        \item Diabetes
        \item Age
    \end{itemize}
\end{itemize}

\subsection{Model Performance}
The models were evaluated using multiple metrics:
\begin{itemize}
    \item Accuracy
    \item Precision
    \item Recall
    \item F1-score
    \item AUC-ROC
\end{itemize}

\section{Public Health Implications}
\subsection{Key Recommendations}
\begin{enumerate}
    \item Implement targeted screening programs for individuals with multiple risk factors
    \item Focus on preventive measures for modifiable risk factors
    \item Develop early intervention strategies based on predictive modeling
\end{enumerate}

\subsection{Limitations}
\begin{itemize}
    \item Self-reported data may contain inherent biases
    \item Cross-sectional nature of the study limits causal inference
    \item Some important clinical variables may not be captured in the dataset
\end{itemize}



\end{document} 